%%%%%%%%%%%%%%%%%%%%%%%%%%%%%%%%%%%%%%%%%
% Twenty Seconds Resume/CV
% LaTeX Template
% Version 1.0 (14/7/16)
%
% Original author:
% Carmine Spagnuolo (cspagnuolo@unisa.it) with major modifications by 
% Vel (vel@LaTeXTemplates.com) and Harsh (harsh.gadgil@gmail.com)
%
% License:
% The MIT License (see included LICENSE file)
%
%%%%%%%%%%%%%%%%%%%%%%%%%%%%%%%%%%%%%%%%%


%----------------------------------------------------------------------------------------
%	PACKAGES AND OTHER DOCUMENT CONFIGURATIONS
%----------------------------------------------------------------------------------------

\documentclass[letterpaper]{twentysecondcv} % a4paper for A4

% Command for printing skill overview bubbles
\newcommand\skills{ 
~
	\smartdiagram[bubble diagram]{
        \textbf{Internet de las }\\\textbf{Cosas (IoT)},
        \textbf{~~Seguimiento~~}\\\textbf{a conductores},
        \textbf{~Control de~}\\\textbf{~flota~},
        \textbf{Información en }\\\textbf{Tiempo Real},
        \textbf{~Pago~}\\\textbf{Electrónico}
    }
}

% Programming skill bars
\programming{{Software (30\%) / 1.8}, {Hardware (20\% / 1.2}, {Servicios en la nube (50\%) / 3}}

% Projects text
\education{
\textbf{Jaime Hernán Bermeo Ramírez} \\
Ingeniero Electrónico \\
Universidad Surcolombiana \\
2011 - 2017 | Neiva, Huila

\textbf{Jonatan Huergo Aguilar}  \\
Estudiante Ingeniería de Software \\
Universidad Surcolombiana \\
2014 - Actualidad  | Neiva, Huila
}

%----------------------------------------------------------------------------------------
%	 PERSONAL INFORMATION
%----------------------------------------------------------------------------------------
% If you don't need one or more of the below, just remove the content leaving the command, e.g. \cvnumberphone{}

\cvname{IoTransport} % Your name
\cvjobtitle{ Soluciones Inteligentes } % Job
% title/career

\cvlinkedin{}
\cvgithub{Legacier}
\cvnumberphone{(+57) 310 771 8993} % Phone number
\cvsite{iotransport.herokuapp.com} % Personal website
\cvmail{contact.iotransport@gmail.com} % Email address

%----------------------------------------------------------------------------------------

\begin{document}

\makeprofile % Print the sidebar
 
%----------------------------------------------------------------------------------------
%	 EXPERIENCE
%----------------------------------------------------------------------------------------

\section{Qué es IoTransport}

\begin{twenty} % Environment for a list with descriptions
\twentyitem
    	{}
    	{}
        {IoTransport es una iniciativa que busca mejorar el servicio de transporte público tradicional a nivel nacional e incentivar el uso de la bicicleta, por medio de una estrategia que consta de tres pilares.}
        {{}}
        {}
        {\begin{itemize}
        \item Prototipo basado en hardware de bajo costo, que permite a cada vehículo suministrar información en tiempo real sobre la velocidad promedio, cantidad de pasajeros, consumo de combustible, pagos por medio de NFC y el tiempo estimado de llegada por cada punto de parada la ciudad. \\
        \item Plataforma para la supervisión y control de flota en tiempo real. Permite a las empresas de transporte tener estadísticas precisas acerca de la eficacia con la que se está prestando el servicio de transporte público y realizar seguimiento a los conductores para reducir accidentes de tránsito debido a exceso de velocidad, conducción temeraria, entre otros factores que pueden reducir la experiencia del usuario. \\
        \item Aplicación móvil que brinda al usuario información sobre tiempos de viaje, rutas y puntos de parada. Facilita el desplazamiento en el interior de la ciudad, tanto para ciudadanos locales, como para turistas e integra pago electrónico por medio de saldo digital, permitiendo al usuario realizar pagos sin contacto usando un chip NFC, adherido a su teléfono móvil.
        \end{itemize}}
        \\

        
	%\twentyitem{<dates>}{<title>}{<location>}{<description>}
\end{twenty}

%----------------------------------------------------------------------------------------
%	 RESEARCH
%----------------------------------------------------------------------------------------
\section{Visión General}
\begin{twenty}
	\twentyitem
    	{}
		{}
        {}
        {}
        {}
        {
   	    Según cifras del DANE, durante el segundo trimestre del presente año, el transporte urbano de pasajeros en las áreas de cobertura cuenta en promedio con 34148 vehículos en servicio por mes1, que corresponde al 64,9\% de todo el parque automotor. La cantidad de pasajeros que se movilizaron se encuentra alrededor de 417 millones1.\\ \\
   	    {En la actualidad, existe una amplia variedad de soluciones disponibles en el mercado para el conteo de pasajeros y para verificar el estado del tráfico como \textbf{Moovit y Registel} que están implementadas en el SITP de \textbf{Bogotá, Cali y Medellín}. Sin embargo, IoTransport tiene un enfoque distinto, puesto que pretende solventar los principales problemas que conllevan las tecnologías existentes en ciudades donde no existe sistema integrado, pues hasta el momento, en el transporte urbano tradicional no se garantiza que se estén cumpliendo los tiempos de cada ruta en todas las ocasiones, ni que la cantidad de pasajeros en un vehículo está por debajo del máximo permitido.} \\ \\
        {Inicialmente, se pretende implementar la solución en la ciudad de Neiva como prueba piloto, en convenio con las cinco empresas que prestan el servicio con un parque automotor de 711 vehículos2, las cuales son, \textbf{COOMOTOR}, \textbf{COOTRANSHUILA LTDA}, \textbf{FLOTA HUILA S.A.}, \textbf{AUTOBUSES S.A.} y \textbf{COOTRANSNEIVA.} \\ \\
        {Esta información sobre el tráfico se proporcionará a Google mediante su programa de partners de \textbf{Google Transit}, facilitando al ciudadano planificar viajes en el transporte público urbano y mediante la tecnología de Google Maps empleada a nivel mundial, se ofrece información precisa sobre paradas, rutas, horarios y tarifas en ciudades pequeñas y medianas de todo el país, donde los embotellamientos son cada vez más frecuentes.} \\
        }
        }
\end{twenty}

\newpage
\makeprofile

\section{Costo de Inversión}

\begin{twenty}
    \twentyitem
    {}
    {}
    {}
    {}
    {}
    {Para la implementación de toda la solución que contemple IoTransport se requieren las siguientes inversiones \\
    
    \textbf{Para las empresas prestadoras de servicio de transporte público} \\
    \begin{itemize}
        \item Dispositivo: Cada unidad incluye GPS + Transmisión GPRS + Sim CARD + Lector NFC + Caja robusta. \\
        \item instalación de los dispositivos: Para los 711 vehículos de la ciudad de Neiva se requiere una inversión inicial de \$215 Millones de Pesos. \\
        \item Acceso a la plataforma (Mensual): \$300mil + \$20mil por cada vehículo presente en la plataforma. Incluye 10Mb de datos mensual con empresa de telefonía para transmisión de información. \\ \\
    \end{itemize}
    
    \textbf{Gastos internos de la empresa} \\
    \begin{itemize}
        \item Desarrollador: Salario de \$2.5 Millones / Mes \\
        \item Diseño gráfico: Salario de \$1.2 Millones / Mes \\
        \item 2 técnicos: Salario de \$781.242 / Mes \\
        \item Dominio web empresarial: \$60mil / año \\
        \item Servidor Privado Virtual (VPS): \$600 USD / año \\
    \end{itemize}
    
    \textbf{Estimacion de Ganancias Anual} \\
    \begin{itemize}
        \item Suscripción a IoTransport (Plataforma): \$170 Millones Aprox.
    \end{itemize}
    
    }
    
 
\end{twenty}


\section{Fuentes}
Boletín técnico. (2018). Encuesta de Transporte Urbano de Pasajeros ETUP.\\ https://www.dane.gov.co/files/investigaciones/boletines/transporte/bol\_transp \\ \_IItrim18.pdf \\

Alcaldía de Neiva. (2018). Directorio parque automotor Neiva.\\ http://www.alcaldianeiva.gov.co/NuestraAlcaldia/Otros\%20Directorios/Forms/ \\ DispForm.aspx?ID=10 \\

Moovit. (Actual). Aplicación de transporte público.\\ Sitio web 
https://www.company.moovit.com/ \\

Registel Colombia. (Actual)\\ Sitio web http://registelcolombia.com \\

Google Transit Partners Program. (Actual) \\
Sitio web http://maps.google.com/help/maps/mapcontent/transit/

\end{document} 
